\chapter{Introdução}
\label{sec_introducao}

\section{Contextualização}

Atualmente, nas organizações contemporâneas, a informação e a forma como ela é transmitida tornaram-se um dos ativos mais valiosos. Segundo \cite{turban2010}, a informação é essencial para apoiar decisões estratégicas e operacionais. A gestão eficiente desse recurso depende, em grande parte, da capacidade dos sistemas corporativos e dos colaboradores em integrar dados, informações, processos e pessoas, reforçando o que afirmam \cite{laudon2020} ao destacar que sistemas fragmentados prejudicam o desempenho organizacional.

Entretanto, muitas empresas ainda operam com sistemas legados e fragmentados, desenvolvidos em épocas em que a estrutura tecnológica e os métodos de integração não eram adequados, o que dificulta a comunicação entre sistemas. Esse cenário resulta em retrabalho, desperdício de tempo, inconsistências e maior dificuldade na tomada de decisão, conforme apontam \cite{stair2018}.

No setor de energia elétrica, como é o caso da empresa Light, a dependência de sistemas integrados é ainda mais evidente. Operações de campo, monitoramento de equipes, gestão de contratos e atendimento ao cliente exigem soluções tecnológicas capazes de consolidar informações em tempo real e oferecer suporte confiável às rotinas críticas, como defendem \cite{rezende2017}.

Especificamente na área de Compra de Energia, podem ser citados os contratos do ACR (Ambiente de Contratação Regulada) e do ACL (Ambiente de Contratação Livre), que dependem de informações provenientes de diferentes sistemas. Um exemplo disso é a CCEE (Câmara de Comercialização de Energia Elétrica), na qual há o ingresso de notas em outro sistema cuja base de dados ainda não é automatizada nem integrada aos demais sistemas corporativos — um problema recorrente em organizações que carecem de integração plena, como discutido por \cite{pressman2016}.

\section{Problema Investigado}

Diante desse cenário, identifica-se que, com o avanço dos mercados ACR e ACL, os benefícios e impactos da unificação de sistemas se tornam cada vez mais relevantes, especialmente considerando o aumento do volume de dados e de clientes. Para \cite{davenport2018}, sistemas integrados promovem inteligência operacional e reduzem gargalos informacionais. Assim, soluções unificadas, como o aplicativo \textit{InLight}, tornam-se essenciais para otimizar processos, reduzir erros e centralizar informações estratégicas.

\section{Motivação}

A escolha do tema está relacionada ao atual cenário de transformação digital, no qual a integração e a modernização de sistemas são estratégias fundamentais para ganhos de eficiência. Segundo \cite{schwab2016}, a digitalização intensifica a necessidade de automação e integração de processos. No caso da Light, que abrange diversas áreas e sistemas corporativos, a comunicação eficaz e a padronização das informações são fatores críticos para o bom desempenho das operações \cite{laudon2020}.

Assim, este estudo se justifica tanto como contribuição prática à organização quanto como exercício acadêmico de aplicação dos conhecimentos adquiridos no curso de Sistemas de Informação.

\section{Objetivos}

O objetivo geral deste trabalho é analisar os benefícios e impactos da unificação de sistemas corporativos, a partir de uma abordagem teórica e de uma análise da possível da implementação do aplicativo \textit{InLight}.

\textbf{Objetivos específicos:}
\begin{itemize}
    \item Mapear os principais desafios e limitações decorrentes da fragmentação de sistemas;
    \item Revisar literatura acadêmica e técnica sobre unificação tecnológica e seus efeitos organizacionais;
    \item Descrever como seria o processo de implementação do aplicativo \textit{InLight} na empresa Light junto com seus impactos;
    \item Mensurar os possíveis resultados que podem ser obtidos após a adoção do aplicativo, considerando indicadores de eficiência, integração e usabilidade;
    \item Identificar oportunidades de melhoria e perspectivas futuras para sistemas unificados em ambientes corporativos.
\end{itemize}

\section{Justificativa}

A unificação de sistemas corporativos representa um passo fundamental para organizações que buscam maior eficiência, confiabilidade e segurança na gestão da informação. Segundo \cite{rezende2017}, a integração tecnológica reduz redundâncias, melhora o fluxo de informações e fortalece a governança organizacional. No contexto da Light, tal integração possibilita o aperfeiçoamento dos fluxos de trabalho, a redução de retrabalho e a melhoria de processos críticos.

Portanto, o estudo contribui tanto para o aprimoramento das práticas empresariais quanto para o avanço das discussões acadêmicas sobre sistemas integrados e transformação digital \cite{davenport2018}.

\section{Questões de Pesquisa para o Estudo de Caso}

A partir do problema apresentado, este Trabalho de Conclusão de Curso busca responder às seguintes questões:
\begin{itemize}
    \item Como sistemas fragmentados impactam a eficiência operacional da empresa?
    \item De que forma a unificação de sistemas, por meio do aplicativo InLight, contribui para a integração de processos internos e a melhoria do fluxo de informações?
    \item Como observar e mensurar os benefícios e limitações decorrentes da unificação de sistemas no ambiente corporativo?
    \item Como a visão dos usuários e gestores influencia a adoção de sistemas unificados?
\end{itemize}

\section{Delimitação do Estudo}

Este trabalho visa à análise teórica e prática da unificação de sistemas corporativos, com foco na implementação de sistemas, adequação a novas tecnologias e seus impactos. Como orienta \cite{yin2015}, estudos de caso permitem investigar fenômenos contemporâneos dentro de seu contexto real, justificando a abordagem adotada.

A coleta de dados será realizada com base em documentos institucionais disponibilizados, sem acesso a informações confidenciais, preservando o sigilo e a ética profissional, conforme recomenda \cite{gil2019}.

\section{Estrutura do Estudo}

O trabalho está estruturado em seis capítulos, conforme descrito a seguir:

\begin{itemize}
    \item \textbf{Capítulo 1 - Introdução:} apresenta a contextualização do tema, problema investigado, motivação, objetivos, justificativa, questões de pesquisa, delimitação do estudo e estrutura geral;
    \item \textbf{Capítulo 2 - Fundamentação Teórica:} reúne conceitos e definições sobre sistemas de informação, unificação com novas tecnologias, integração de processos e impactos dentro de uma organização;
    \item \textbf{Capítulo 3 - Revisão da Literatura:} sistematiza pesquisas acadêmicas e técnicas que abordam o tema em questão, destacando autores e estudos relevantes para o embasamento teórico, trazendo pontos positivos e negativos sobre a unificação e suas aplicações;
    \item \textbf{Capítulo 4 - Estudo de Caso: Aplicativo InLight:} descreve o processo de concepção, desenvolvimento e implementação do aplicativo na empresa Light, além de como vem modificando o dia a dia da empresa;
    \item \textbf{Capítulo 5 - Análise dos Resultados:} apresenta a comparação entre os cenários pré e pós-implementação, destacando ganhos de eficiência, integração e percepção dos usuários;
    \item \textbf{Capítulo 6 - Conclusões e Recomendações:} desfecho final do nosso estudo, discutindo o contexto geral abordado além de sugestões futuras tanto no contexto Light quanto sobre unificação de sistemas.
\end{itemize}
