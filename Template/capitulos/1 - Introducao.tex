\chapter{Introdução}
\label{sec_introducao}

\section{Contextualização}

Atualmente, nas organizações contemporâneas, a informação e a forma como ela é transmitida tornaram-se um dos ativos mais valiosos. A gestão eficiente desse recurso depende, em grande parte, da capacidade dos sistemas corporativos e dos colaboradores em integrar dados, informações, processos e pessoas. Entretanto, muitas empresas ainda operam com sistemas fragmentados, desenvolvidos em épocas em que a estrutura tecnológica e os métodos de integração não eram adequados, o que dificulta a comunicação entre os sistemas. Esse cenário resulta em retrabalho, desperdício de tempo, inconsistências e maior dificuldade na tomada de decisão.

No setor de energia elétrica, como é o caso da empresa Light, a dependência de sistemas integrados é ainda mais evidente. Operações de campo, monitoramento de equipes, gestão de contratos e atendimento ao cliente exigem soluções tecnológicas capazes de consolidar informações em tempo real e oferecer suporte seguro e ágil às rotinas críticas.

Especificamente na área de compra de energia, podem ser citados os contratos do ACR (Ambiente de Contratação Regulada) e do ACL (Ambiente de Contratação Livre), que dependem de informações provenientes de diferentes sistemas. Um exemplo disso é a CCEE (Câmara de Comercialização de Energia Elétrica), na qual há o ingresso de notas em outro sistema, cuja base de dados ainda não é automatizada nem integrada aos demais sistemas corporativos.

\section{Problema Investigado}

Diante desse cenário, identifica-se que, com o avanço dos mercados ACR e ACL, os benefícios e impactos da unificação de sistemas são cada vez mais relevantes, especialmente considerando o aumento do volume de dados e de clientes em ambos os ambientes. Assim, soluções integradas, como o aplicativo \textit{InLight}, tornam-se essenciais para otimizar processos, reduzir erros e centralizar informações estratégicas.

\section{Motivação}

A escolha do tema está relacionada ao contexto atual de transformação digital, no qual a integração e a modernização de sistemas buscam otimizar recursos, tempo e processos, promovendo eficiência operacional e maior qualidade na prestação de serviços. No caso específico da Light, que abrange diversas áreas e sistemas corporativos, a comunicação eficaz e a padronização das informações são fatores críticos para o bom desempenho das operações. Assim, este estudo se justifica tanto como contribuição prática à organização quanto como exercício acadêmico de aplicação dos conhecimentos adquiridos no curso de Sistemas de Informação.

\section{Objetivos}

O objetivo geral deste trabalho é analisar os benefícios e impactos da unificação de sistemas corporativos, a partir de uma abordagem teórica e de uma análise prática da implementação do aplicativo \textit{InLight}.

\textbf{Objetivos específicos:}
\begin{itemize}
    \item Mapear os principais desafios e limitações decorrentes da fragmentação de sistemas;
    \item Revisar literatura acadêmica e técnica sobre unificação tecnológica e seus efeitos organizacionais;
    \item Descrever o processo de implementação do aplicativo \textit{InLight} na empresa Light;
    \item Mensurar os resultados obtidos após a adoção do aplicativo, considerando indicadores de eficiência, integração e usabilidade;
    \item Identificar oportunidades de melhoria e perspectivas futuras para sistemas unificados em ambientes corporativos.
\end{itemize}

\section{Justificativa}

A unificação de sistemas corporativos representa um passo fundamental para organizações que buscam maior eficiência, confiabilidade e segurança na gestão da informação. No contexto da Light, tal integração possibilita o aperfeiçoamento dos fluxos de trabalho, a redução de retrabalho e o fortalecimento da governança de dados. Portanto, o estudo contribui tanto para o aprimoramento das práticas empresariais quanto para o avanço das discussões acadêmicas sobre sistemas integrados e transformação digital no setor elétrico.

\section{Questões de Pesquisa para o Estudo de Caso}

A partir do problema em foco apresentado, o Trabalho de Conclusão de Curso vem por meio deste responder às seguintes questões de pesquisa:
\begin{itemize}
    \item Como sistemas fragmentados em sistema corporativos impactam na eficiência operacional da empresa?
    \item De que forma a unificação de sistemas, por meio do aplicativo InLight, contribui para a integração de processos internos e a melhora o fluxo de informações?
    \item Como podemos observar e mensurar os benefícios e limitações que a unificação de sistemas acaba acarretando no ambiente corporativo?
    \item Como a visão dos usuários e gestores influencia na adoção de sistemas unificados?
\end{itemize}
Realizando este estudo de caso, pode-se assim direcionar a análise dos dados, que permite relacionar a teoria sobre unificação de sistemas e prática abordando o contexto da empresa LIGHT.
\section{Delimitação do Estudo}

Este trabalho, visa a análise teórica e prática da unificação de sistemas corporativos, com foco na implementação de sistemas, adequação a novas tecnologias, seus benefícios e malefícios.
Abordando a empresa já citada, a pesquisa não se propões a abranger todas as áreas tecnológicas da organização, abordando no escopo a integração do sistema InLight e demais melhorias que possam ser abordadas referindo-se ao ACR e ACL.
Além disso, a coleta de dados será realizada com base em dados e documentos institucionais disponibilizados, sem acesso a informações confidenciais, preservando o sigilo e a ética profissional.

\section{Estrutura do Estudo}
O trabalho está estruturado em seis capítulos, conforme descrito a seguir:

\begin{itemize}
    \item \textbf{Capítulo 1 – Introdução:} apresenta a contextualização do tema, problema investigado, motivação, objetivos, justificativa, questões de pesquisa, delimitação do estudo e estrutura geral;
    \item \textbf{Capítulo 2 – Fundamentação Teórica:} reúne conceitos e definições sobre sistemas de informação, unificação com novas tecnologias, integração de processos e impactos dentro de uma organização;
    \item \textbf{Capítulo 3 – Revisão da Literatura:} sistematiza pesquisas acadêmicas e técnicas que abordam o tema em questão, destacando autores e estudos relevantes para o embasamento teórico, trazendo pontos positivos e negativos sobre a unificação e suas aplicações;
    \item \textbf{Capítulo 4 – Estudo de Caso: Aplicativo InLight:} descreve o processo de concepção, desenvolvimento e implementação do aplicativo na empresa LIGHT, além de como vem modificando o dia a dia da empresa;
    \item \textbf{Capítulo 5 – Análise dos Resultados:} apresenta a comparação entre os cenários pré e pós-implementação, destacando ganhos de eficiência, integração e percepção dos usuários;
    \item \textbf{Capítulo 6 – Conclusões e Recomendações:} desfecho final do nosso estudo, discutindo o contexto geral abordado além de sugestões futuras tanto no contexto LIGHT quanto sobre unificação de sistemas.
\end{itemize}
